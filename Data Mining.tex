\documentclass[conference]{IEEEtran}
\IEEEoverridecommandlockouts
% The preceding line is only needed to identify funding in the first footnote. If that is unneeded, please comment it out.
\usepackage{cite}
\usepackage{amsmath,amssymb,amsfonts}
\usepackage{algorithmic}
\usepackage{graphicx}
\usepackage{textcomp}
\usepackage{xcolor}
\def\BibTeX{{\rm B\kern-.05em{\sc i\kern-.025em b}\kern-.08em
    T\kern-.1667em\lower.7ex\hbox{E}\kern-.125emX}}
\begin{document}

\title{Recommender systems}
% {\footnotesize \textsuperscript{*}Note: Sub-titles are not captured in Xplore and
% should not be used}
% \thanks{Vishwanath P}
% }

\author{\IEEEauthorblockN{Anubhav Ujjawal}
\IEEEauthorblockA{\textit{Computer Science and Engineering} \\
\textit{IIIT Sri City}\\
Sri City, India \\
anubhav.u16@iiits.in}
\and
\IEEEauthorblockN{Chandrajeet Choudhary}
\IEEEauthorblockA{\textit{Computer Science and Engineering} \\
\textit{IIIT Sri City}\\
Sri City, India \\
chandrajeet.c16@iiits.in}
}

\maketitle

\begin{abstract}
A Recommender system or a recommendation system is a subclass of information filtering system that seeks to predict the ``rating" or ``preference" a user would give to an item.
We discuss various methods used to create recommender systems, and compare their pros and cons.
\end{abstract}

\begin{IEEEkeywords}
Recommender Systems, Collaborative filtering, Content-based filtering
\end{IEEEkeywords}

\section{Introduction}
A Recommender system or a recommendation system is a subclass of information filtering system that seeks to predict the ``rating" or ``preference" a user would give to an item. Recommender systems are utilized in a variety of areas including movies, music, news, books, research articles, search queries, social tags, and products in general.
Recommender systems depend heavily on the feedback data provided by the users and the field for which they are built.
Typically, a recommendation is provided in one of two ways- \textit{Collaborative filtering (CF)} or \textit{Content-based filtering (CB)}. Collaborative filtering is based on the assumption that people who agreed in the past will agree in the future, and that they will like similar kinds of items as they liked in the past.
However, users tend to drift from their behavior, and this may lead to unwanted results. Content-based filtering methods are based on a description of the item and a profile of the user’s preferences. It is domain dependent, i.e, when the system is limited to recommending content of the same type as the user is already using, the value from the recommendation system is significantly less than when other content types from other services can be recommended.
Very often these methods are combined to create a \textit{Hybrid model}. A lot of work has been done on creating finely tuned models for recommendation systems, the most significant one being the Netflix prize, which was won by \textit{BellKor's Pragmatic Chaos}.

\section{Collaborative filtering}
Collaborative filtering algorithms are based on the idea that if two users have a similar rating history, they will behave similarly in the future.  This is an important method because it doesn't depends on the other additional info about the items to produce recommendations.

\subsection{Memory Based Techniques}
\subsubsection{User-User based collaborative filtering}
Let us assume a collection of users \(U\) with \(i^{th}\) user and rating vector of \(ith\) user being  \(u_{i}\) and $r_{i} (r_{i0}, r_{i1}, r_{i2}, ...., r_{in})$ where 0, 1, 2, ... n, are item ids respectively. If a rating $r_{ik}$ is missing, we leave it blank. We find a set of users similar to \(u_{i}\) and call them neighbourhood of \(u_{i}\). We then proceed to find other items liked by a lot of users in the neighbourhood of \(u_{i}\) and proceed to recommend them to user \(u_{i}\).

To find the neighbourhood of any user \(u_{i}\), we need to define the similarity metric to calculate similarity between \(u_{i}\) and \(u_{j}\). The \textit{Cosine similarity} calculates the similarity between two rating vectors \(r_i\) and \(r_j\) as:
\begin{equation}
    cos(r_i, r_j) = \frac{r_i\boldsymbol{\cdot}r_j}{\lvert r_i \rvert \lvert r_j \rvert }
\end{equation}
However, the problem with cosine similarity is that is treats the missing ratings as negative i.e, if $u_i$ has not rated an item, he didn't like it. Therefore, instead of using cosine similarity we normalize the $r_i$ of a user such that the ratings get centered to 0 and then apply the similarity on this modified rating. This is done by using \textit{Pearson Correlation}.
\begin{equation}
    s(r_i, r_j) = \frac{(r_i - \overline{r_i})\boldsymbol{\cdot}(r_j - \overline{r_j})}{\lvert r_i - \overline{r_i} \rvert \lvert r_j - \overline{r_j} \rvert}
\end{equation}

Here, $\overline{r_i}$ is the mean rating by user $u_i$. While calculating $r_i - \overline{r_i}$, we note that if a rating is missing, we simply mark it as 0 instead of writing $-\overline{r_i}$. \texit{Pearson's Correlation} handles ``tough raters" and ``easy raters" and centers the avg. rating of a user to be 0, i.e, for a user $u_i$, 
\begin{equation}
\sum_{j=0}^n r_{ij} -  \overline{r_{i}} = 0
\end{equation}
The next step is to identify the items the users in neighbourhood of $u_i$ liked, then remove the items $u_i$ he has interacted with. We know that a better correlation score of $u_i$ with $u_j$ than with $u_k$ mean $u_i$ is more similar to $u_j$ than $u_k$, and we take this into account while recommending the items to $u_i$. 

For any $p_{th}$ item not interacted by $u_i$, we first select the top users from the neighbourhood of $u_i$ (N) who interacted with the item, and then calculating the expected rating using the following equation.
\begin{equation}
    r_{ip} = \frac{\sum_{y \epsilon N} s(u_i, u_y)r_{yp}}{\sum_{y \epsilon N} s(u_i, u_y)}
\end{equation}

Now we have the rating, so we can recommend $p_{th}$ item to $u_i$ if rating is above a certain threshold.
\subsubsection{Item-Item based collaborative filtering}
In item-item based collaborative filtering, for the $i^{th}$ item $it_{i}$, we estimate the rating based on other similar items. We use the same similarity metrics and prediction in user-user filtering. The roles change, i.e, now we have a rating vector for each item $it_i(it_{i0}, it_{i1}, it_{i2}, ... it_{in})$ where 0, 1, 2, ... n are user ids.

Note that \textit{user-user filtering} and \textit{item-item filtering} are dual approaches, i.e, they yield the same result, theoretically. However, item-item hugely outperforms user-user based collaborative filtering. This is due to the fact that items belong to a small set of genres, while users have varied taste. Also, item similarity is more meaningful than user similarity. 

\subsection{Model Based Techniques}
In \textit{Model Based Techniques}, the ratings are used to implement a model that will improve the results of the collaborative filtering in order to find patterns in the data. To build a model some data mining or machine learning algorithms is applied. These kinds of models are pretty useful to recommend a set of items, in the fastest way and show similar results to the Memory-based models. Model-based techniques are based on \textit{Matrix factorization (MF)}, an unsupervised learning method for dimensionality reduction. This family of methods became widely known during the Netflix prize challenge due to its effectiveness as reported by Simon Funk in his 2006 blog post. Basically, \textit{MF} learns the latent preferences of users and items from the ratings in order to make a prediction of the missing ratings, using the dot product of the latent features of users and items.
Some of the techniques that might be applied are based on Dimensionality Reduction techniques, for instance, \textit{Principal Component Analysis (PCA), Singular Value Decomposition (SVD), Probabilistic Matrix Factorization (PMF), Matrix completion Technique, Latent Semantic methods, and Regression and Clustering}. We discuss the 2 most popular techniques namely, \textit{PCA, SVD}.

\subsubsection{Principal Component Analysis (PCA)}
The principal component analysis is known by using an orthogonal transformation, since it makes use of the eigen vectors of the covariance matrix. The idea is to transform a set of variables that might be correlated, into a set of new uncorrelated vectors.These new vectors are named the principal components.
Given that the main purpose is to reduce dimensions, the set of original variables is greater than the final number of principal components. However, when
 we reduce dimensions, we also lose some information, but the construction of this methodology allows the retain the maximal variance and the least squared errors are minimized. Each component retains a percentage of the variance, being the first component the one that retains the most, and the percentage retained starts to decrease in each component. Then the dimensions can be reduced by deciding the amount of variance we want to keep.
 
 \subsubsection{Singular Value Decomposition (SVD)}
 The most popular approach is \textit{Singular value decomposition (SVD)}. The general equation can be expressed as $X = U\times S \times V^t$. Given an $n \times m$ matrix $X$, then $U$ is an $r \times r$ diagonal matrix with non-negative real numbers on the diagonal, and $V^t$ is an $r \times n$ orthogonal matrix. The elements on the diagonal S are known as the singular values of X.
 
 Then the user-item matrix defined here as $X$ can be expressed as a composition of $U$, $S$ and $V$. Where $U$ is representing the feature vectors corresponding to the users in the hidden feature space and $V$ is representing the feature vectors corresponding to the items in the hidden feature space.
 
 Now we can make a prediction by multiplying the matrices $U$, $S$ and $V^t$. That is
to say,$$X = U \times S \times V^t$$

\subsection{Implementation discussion}
\textit{Memory Based Techniques} are very useful, because they are simple to apply and the results are efficient enough, since they produce good results in most of the cases. However, \textit{sparsity} of the User-Item matrix and \textit{scalability} (nearest neighbor requires computation that grows with both the number of users and the number of items) are some major issues with Memory Based Techniques.

However, the \textit{Model-based techniques} are based on Matrix factorization and can deal better with scalability and sparsity than Memory-based CF. These techniques try to find a relation between the items in the user item matrix using the latent preferences, and then make a comparison in the top-N recommendations. MF is highly prone to over-fitting and their approaches can be very slow and computationally expensive.

\section{Content-based filtering}

% \subsection{Maintaining the Integrity of the Specifications}

Content-based filtering, also referred to as cognitive filtering, recommends items based on a comparison between the content of the items and a user profile. The main difference between this approach and
the \textit{Collaborative filtering (CF)} is that \textit{Content-based filtering (CB)} offers the recommendation based not only similarity in user history, but it is more about the information from the products. To implement this methodology, it is necessary to have information describing each item, and some kind of user profile describing what the user likes. The main task is to learn user preferences and then recommend items that are similar to user preferences.

Generally, \textit{Content-based filtering (CB)} techniques are applied to recommend text documents, for example, web pages or newspapers. However, the most important part is that the content of the item is text description, including text documents so we need a structured data to describe each item in the form of some feature vector \textit{y}.

The core of this approach is to create a user's preferences model based on those feature vectors. There are several techniques to develop such models. For instance \textit{Term Frequency (TF) or Inverse Document Frequency (IDF)} and, \texit{Probabilistic methods (\textit{Naive Bayes}, \textit{Support Vector Machine (SVM)}, \textit{Decision Trees})}. In the following section, a description will be given for each approach.

\subsection{Term Frequency (TF) or Inverse Document Frequency (IDF)}\label{AA}

Information retrieval and Text mining usually make use of TF-IDF weights to determine the importance of a word in a text or a document in a corpus. The importance is highly correlated to the popularity of the word in text, it decreases its value with the presence of the word in the text or corpus. The more popular words do not give us extra information to recognize relevant documents in the corpus.

Let us assume that there are \(N\) documents which can be recommended, \(k_{i}\) is the keyword that is present in \(n_{i}\) documents. Now, the number of times \(k_{i}\) is in document \(d_{i}\) is defined as \(f_{ij}\). Then,
\begin{equation}
    TF_{i,j} = \frac{f_{ij}}{max_{z}f_{z,j}}
\end{equation}
Where \(TF_{i,j}\) is normalized term frequency of keyword \(k_{i}\) in document \(d_{j}\). 

And, the inverse document frequency (IDF) for a keyword \(k_{i}\) is defined as
\begin{equation}
    IDF_{i} = log\frac{N}{n_{i}}
\end{equation}
Where \(N\) is total documents and \(n_{i}\) is number of documents containing keyword \(k_{i}\).

TF-IDF weight for keyword \(k_{i}\) in document \(d_{i}\) is as
\begin{equation}
    w_{i,j} = {TF_{i,j}} \times {IDF_{i}}
\end{equation}

\subsection{Probabilistic Methods}\label{BB}
The probabilistic methods are to determine the probability that a user \(u_{i}\) will be interested in the item \(p_{j}\). The probability estimation is done on the basis of the user-item rating matrix (S).

Recommendations made by these probabilistic methods do not require the user profile to recommend items.
\section{Prepare Your Paper Before Styling}
Before you begin to format your paper, first write and save the content as a 
separate text file. Complete all content and organizational editing before 
formatting. Please note sections \ref{AA}--\ref{SCM} below for more information on 
proofreading, spelling and grammar.

Keep your text and graphic files separate until after the text has been 
formatted and styled. Do not number text heads---{\LaTeX} will do that 
for you.

\subsection{Abbreviations and Acronyms}\label{AA}
Define abbreviations and acronyms the first time they are used in the text, 
even after they have been defined in the abstract. Abbreviations such as 
IEEE, SI, MKS, CGS, ac, dc, and rms do not have to be defined. Do not use 
abbreviations in the title or heads unless they are unavoidable.

\subsection{Units}
\begin{itemize}
\item Use either SI (MKS) or CGS as primary units. (SI units are encouraged.) English units may be used as secondary units (in parentheses). An exception would be the use of English units as identifiers in trade, such as ``3.5-inch disk drive''.
\item Avoid combining SI and CGS units, such as current in amperes and magnetic field in oersteds. This often leads to confusion because equations do not balance dimensionally. If you must use mixed units, clearly state the units for each quantity that you use in an equation.
\item Do not mix complete spellings and abbreviations of units: ``Wb/m\textsuperscript{2}'' or ``webers per square meter'', not ``webers/m\textsuperscript{2}''. Spell out units when they appear in text: ``. . . a few henries'', not ``. . . a few H''.
\item Use a zero before decimal points: ``0.25'', not ``.25''. Use ``cm\textsuperscript{3}'', not ``cc''.)
\end{itemize}

\subsection{Equations}
Number equations consecutively. To make your 
equations more compact, you may use the solidus (~/~), the exp function, or 
appropriate exponents. Italicize Roman symbols for quantities and variables, 
but not Greek symbols. Use a long dash rather than a hyphen for a minus 
sign. Punctuate equations with commas or periods when they are part of a 
sentence, as in:
\begin{equation}
a+b=\gamma\label{eq}
\end{equation}

Be sure that the 
symbols in your equation have been defined before or immediately following 
the equation. Use ``\eqref{eq}'', not ``Eq.~\eqref{eq}'' or ``equation \eqref{eq}'', except at 
the beginning of a sentence: ``Equation \eqref{eq} is . . .''

\subsection{\LaTeX-Specific Advice}

Please use ``soft'' (e.g., \verb|\eqref{Eq}|) cross references instead
of ``hard'' references (e.g., \verb|(1)|). That will make it possible
to combine sections, add equations, or change the order of figures or
citations without having to go through the file line by line.

Please don't use the \verb|{eqnarray}| equation environment. Use
\verb|{align}| or \verb|{IEEEeqnarray}| instead. The \verb|{eqnarray}|
environment leaves unsightly spaces around relation symbols.

Please note that the \verb|{subequations}| environment in {\LaTeX}
will increment the main equation counter even when there are no
equation numbers displayed. If you forget that, you might write an
article in which the equation numbers skip from (17) to (20), causing
the copy editors to wonder if you've discovered a new method of
counting.

{\BibTeX} does not work by magic. It doesn't get the bibliographic
data from thin air but from .bib files. If you use {\BibTeX} to produce a
bibliography you must send the .bib files. 

{\LaTeX} can't read your mind. If you assign the same label to a
subsubsection and a table, you might find that Table I has been cross
referenced as Table IV-B3. 

{\LaTeX} does not have precognitive abilities. If you put a
\verb|\label| command before the command that updates the counter it's
supposed to be using, the label will pick up the last counter to be
cross referenced instead. In particular, a \verb|\label| command
should not go before the caption of a figure or a table.

Do not use \verb|\nonumber| inside the \verb|{array}| environment. It
will not stop equation numbers inside \verb|{array}| (there won't be
any anyway) and it might stop a wanted equation number in the
surrounding equation.

\subsection{Some Common Mistakes}\label{SCM}
\begin{itemize}
\item The word ``data'' is plural, not singular.
\item The subscript for the permeability of vacuum $\mu_{0}$, and other common scientific constants, is zero with subscript formatting, not a lowercase letter ``o''.
\item In American English, commas, semicolons, periods, question and exclamation marks are located within quotation marks only when a complete thought or name is cited, such as a title or full quotation. When quotation marks are used, instead of a bold or italic typeface, to highlight a word or phrase, punctuation should appear outside of the quotation marks. A parenthetical phrase or statement at the end of a sentence is punctuated outside of the closing parenthesis (like this). (A parenthetical sentence is punctuated within the parentheses.)
\item A graph within a graph is an ``inset'', not an ``insert''. The word alternatively is preferred to the word ``alternately'' (unless you really mean something that alternates).
\item Do not use the word ``essentially'' to mean ``approximately'' or ``effectively''.
\item In your paper title, if the words ``that uses'' can accurately replace the word ``using'', capitalize the ``u''; if not, keep using lower-cased.
\item Be aware of the different meanings of the homophones ``affect'' and ``effect'', ``complement'' and ``compliment'', ``discreet'' and ``discrete'', ``principal'' and ``principle''.
\item Do not confuse ``imply'' and ``infer''.
\item The prefix ``non'' is not a word; it should be joined to the word it modifies, usually without a hyphen.
\item There is no period after the ``et'' in the Latin abbreviation ``et al.''.
\item The abbreviation ``i.e.'' means ``that is'', and the abbreviation ``e.g.'' means ``for example''.
\end{itemize}
An excellent style manual for science writers is \cite{b7}.

\subsection{Authors and Affiliations}
\textbf{The class file is designed for, but not limited to, six authors.} A 
minimum of one author is required for all conference articles. Author names 
should be listed starting from left to right and then moving down to the 
next line. This is the author sequence that will be used in future citations 
and by indexing services. Names should not be listed in columns nor group by 
affiliation. Please keep your affiliations as succinct as possible (for 
example, do not differentiate among departments of the same organization).

\subsection{Identify the Headings}
Headings, or heads, are organizational devices that guide the reader through 
your paper. There are two types: component heads and text heads.

Component heads identify the different components of your paper and are not 
topically subordinate to each other. Examples include Acknowledgments and 
References and, for these, the correct style to use is ``Heading 5''. Use 
``figure caption'' for your Figure captions, and ``table head'' for your 
table title. Run-in heads, such as ``Abstract'', will require you to apply a 
style (in this case, italic) in addition to the style provided by the drop 
down menu to differentiate the head from the text.

Text heads organize the topics on a relational, hierarchical basis. For 
example, the paper title is the primary text head because all subsequent 
material relates and elaborates on this one topic. If there are two or more 
sub-topics, the next level head (uppercase Roman numerals) should be used 
and, conversely, if there are not at least two sub-topics, then no subheads 
should be introduced.

\subsection{Figures and Tables}
\paragraph{Positioning Figures and Tables} Place figures and tables at the top and 
bottom of columns. Avoid placing them in the middle of columns. Large 
figures and tables may span across both columns. Figure captions should be 
below the figures; table heads should appear above the tables. Insert 
figures and tables after they are cited in the text. Use the abbreviation 
``Fig.~\ref{fig}'', even at the beginning of a sentence.

\begin{table}[htbp]
\caption{Table Type Styles}
\begin{center}
\begin{tabular}{|c|c|c|c|}
\hline
\textbf{Table}&\multicolumn{3}{|c|}{\textbf{Table Column Head}} \\
\cline{2-4} 
\textbf{Head} & \textbf{\textit{Table column subhead}}& \textbf{\textit{Subhead}}& \textbf{\textit{Subhead}} \\
\hline
copy& More table copy$^{\mathrm{a}}$& &  \\
\hline
\multicolumn{4}{l}{$^{\mathrm{a}}$Sample of a Table footnote.}
\end{tabular}
\label{tab1}
\end{center}
\end{table}

\begin{figure}[htbp]
\centerline{\includegraphics{fig1.png}}
\caption{Example of a figure caption.}
\label{fig}
\end{figure}

Figure Labels: Use 8 point Times New Roman for Figure labels. Use words 
rather than symbols or abbreviations when writing Figure axis labels to 
avoid confusing the reader. As an example, write the quantity 
``Magnetization'', or ``Magnetization, M'', not just ``M''. If including 
units in the label, present them within parentheses. Do not label axes only 
with units. In the example, write ``Magnetization (A/m)'' or ``Magnetization 
\{A[m(1)]\}'', not just ``A/m''. Do not label axes with a ratio of 
quantities and units. For example, write ``Temperature (K)'', not 
``Temperature/K''.

\section*{Acknowledgment}

The preferred spelling of the word ``acknowledgment'' in America is without 
an ``e'' after the ``g''. Avoid the stilted expression ``one of us (R. B. 
G.) thanks $\ldots$''. Instead, try ``R. B. G. thanks$\ldots$''. Put sponsor 
acknowledgments in the unnumbered footnote on the first page.

\section*{References}

Please number citations consecutively within brackets \cite{b1}. The 
sentence punctuation follows the bracket \cite{b2}. Refer simply to the reference 
number, as in \cite{b3}---do not use ``Ref. \cite{b3}'' or ``reference \cite{b3}'' except at 
the beginning of a sentence: ``Reference \cite{b3} was the first $\ldots$''

Number footnotes separately in superscripts. Place the actual footnote at 
the bottom of the column in which it was cited. Do not put footnotes in the 
abstract or reference list. Use letters for table footnotes.

Unless there are six authors or more give all authors' names; do not use 
``et al.''. Papers that have not been published, even if they have been 
submitted for publication, should be cited as ``unpublished'' \cite{b4}. Papers 
that have been accepted for publication should be cited as ``in press'' \cite{b5}. 
Capitalize only the first word in a paper title, except for proper nouns and 
element symbols.

For papers published in translation journals, please give the English 
citation first, followed by the original foreign-language citation \cite{b6}.

\begin{thebibliography}{00}
\bibitem{b1} G. Eason, B. Noble, and I. N. Sneddon, ``On certain integrals of Lipschitz-Hankel type involving products of Bessel functions,'' Phil. Trans. Roy. Soc. London, vol. A247, pp. 529--551, April 1955.
\bibitem{b2} J. Clerk Maxwell, A Treatise on Electricity and Magnetism, 3rd ed., vol. 2. Oxford: Clarendon, 1892, pp.68--73.
\bibitem{b3} I. S. Jacobs and C. P. Bean, ``Fine particles, thin films and exchange anisotropy,'' in Magnetism, vol. III, G. T. Rado and H. Suhl, Eds. New York: Academic, 1963, pp. 271--350.
\bibitem{b4} K. Elissa, ``Title of paper if known,'' unpublished.
\bibitem{b5} R. Nicole, ``Title of paper with only first word capitalized,'' J. Name Stand. Abbrev., in press.
\bibitem{b6} Y. Yorozu, M. Hirano, K. Oka, and Y. Tagawa, ``Electron spectroscopy studies on magneto-optical media and plastic substrate interface,'' IEEE Transl. J. Magn. Japan, vol. 2, pp. 740--741, August 1987 [Digests 9th Annual Conf. Magnetics Japan, p. 301, 1982].
\bibitem{b7} M. Young, The Technical Writer's Handbook. Mill Valley, CA: University Science, 1989.
\end{thebibliography}
\vspace{12pt}
\color{red}
IEEE conference templates contain guidance text for composing and formatting conference papers. Please ensure that all template text is removed from your conference paper prior to submission to the conference. Failure to remove the template text from your paper may result in your paper not being published.

\end{document}
